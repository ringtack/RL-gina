\documentclass{article} % For LaTeX2e
\usepackage{rl-gina-report,times}

% Optional math commands from https://github.com/goodfeli/dlbook_notation.
%%%%% NEW MATH DEFINITIONS %%%%%

\usepackage{amsmath,amsfonts,bm}

% Mark sections of captions for referring to divisions of figures
\newcommand{\figleft}{{\em (Left)}}
\newcommand{\figcenter}{{\em (Center)}}
\newcommand{\figright}{{\em (Right)}}
\newcommand{\figtop}{{\em (Top)}}
\newcommand{\figbottom}{{\em (Bottom)}}
\newcommand{\captiona}{{\em (a)}}
\newcommand{\captionb}{{\em (b)}}
\newcommand{\captionc}{{\em (c)}}
\newcommand{\captiond}{{\em (d)}}

% Highlight a newly defined term
\newcommand{\newterm}[1]{{\bf #1}}


% Figure reference, lower-case.
\def\figref#1{figure~\ref{#1}}
% Figure reference, capital. For start of sentence
\def\Figref#1{Figure~\ref{#1}}
\def\twofigref#1#2{figures \ref{#1} and \ref{#2}}
\def\quadfigref#1#2#3#4{figures \ref{#1}, \ref{#2}, \ref{#3} and \ref{#4}}
% Section reference, lower-case.
\def\secref#1{section~\ref{#1}}
% Section reference, capital.
\def\Secref#1{Section~\ref{#1}}
% Reference to two sections.
\def\twosecrefs#1#2{sections \ref{#1} and \ref{#2}}
% Reference to three sections.
\def\secrefs#1#2#3{sections \ref{#1}, \ref{#2} and \ref{#3}}
% Reference to an equation, lower-case.
\def\eqref#1{equation~\ref{#1}}
% Reference to an equation, upper case
\def\Eqref#1{Equation~\ref{#1}}
% A raw reference to an equation---avoid using if possible
\def\plaineqref#1{\ref{#1}}
% Reference to a chapter, lower-case.
\def\chapref#1{chapter~\ref{#1}}
% Reference to an equation, upper case.
\def\Chapref#1{Chapter~\ref{#1}}
% Reference to a range of chapters
\def\rangechapref#1#2{chapters\ref{#1}--\ref{#2}}
% Reference to an algorithm, lower-case.
\def\algref#1{algorithm~\ref{#1}}
% Reference to an algorithm, upper case.
\def\Algref#1{Algorithm~\ref{#1}}
\def\twoalgref#1#2{algorithms \ref{#1} and \ref{#2}}
\def\Twoalgref#1#2{Algorithms \ref{#1} and \ref{#2}}
% Reference to a part, lower case
\def\partref#1{part~\ref{#1}}
% Reference to a part, upper case
\def\Partref#1{Part~\ref{#1}}
\def\twopartref#1#2{parts \ref{#1} and \ref{#2}}

\def\ceil#1{\lceil #1 \rceil}
\def\floor#1{\lfloor #1 \rfloor}
\def\1{\bm{1}}
\newcommand{\train}{\mathcal{D}}
\newcommand{\valid}{\mathcal{D_{\mathrm{valid}}}}
\newcommand{\test}{\mathcal{D_{\mathrm{test}}}}

\def\eps{{\epsilon}}


% Random variables
\def\reta{{\textnormal{$\eta$}}}
\def\ra{{\textnormal{a}}}
\def\rb{{\textnormal{b}}}
\def\rc{{\textnormal{c}}}
\def\rd{{\textnormal{d}}}
\def\re{{\textnormal{e}}}
\def\rf{{\textnormal{f}}}
\def\rg{{\textnormal{g}}}
\def\rh{{\textnormal{h}}}
\def\ri{{\textnormal{i}}}
\def\rj{{\textnormal{j}}}
\def\rk{{\textnormal{k}}}
\def\rl{{\textnormal{l}}}
% rm is already a command, just don't name any random variables m
\def\rn{{\textnormal{n}}}
\def\ro{{\textnormal{o}}}
\def\rp{{\textnormal{p}}}
\def\rq{{\textnormal{q}}}
\def\rr{{\textnormal{r}}}
\def\rs{{\textnormal{s}}}
\def\rt{{\textnormal{t}}}
\def\ru{{\textnormal{u}}}
\def\rv{{\textnormal{v}}}
\def\rw{{\textnormal{w}}}
\def\rx{{\textnormal{x}}}
\def\ry{{\textnormal{y}}}
\def\rz{{\textnormal{z}}}

% Random vectors
\def\rvepsilon{{\mathbf{\epsilon}}}
\def\rvtheta{{\mathbf{\theta}}}
\def\rva{{\mathbf{a}}}
\def\rvb{{\mathbf{b}}}
\def\rvc{{\mathbf{c}}}
\def\rvd{{\mathbf{d}}}
\def\rve{{\mathbf{e}}}
\def\rvf{{\mathbf{f}}}
\def\rvg{{\mathbf{g}}}
\def\rvh{{\mathbf{h}}}
\def\rvu{{\mathbf{i}}}
\def\rvj{{\mathbf{j}}}
\def\rvk{{\mathbf{k}}}
\def\rvl{{\mathbf{l}}}
\def\rvm{{\mathbf{m}}}
\def\rvn{{\mathbf{n}}}
\def\rvo{{\mathbf{o}}}
\def\rvp{{\mathbf{p}}}
\def\rvq{{\mathbf{q}}}
\def\rvr{{\mathbf{r}}}
\def\rvs{{\mathbf{s}}}
\def\rvt{{\mathbf{t}}}
\def\rvu{{\mathbf{u}}}
\def\rvv{{\mathbf{v}}}
\def\rvw{{\mathbf{w}}}
\def\rvx{{\mathbf{x}}}
\def\rvy{{\mathbf{y}}}
\def\rvz{{\mathbf{z}}}

% Elements of random vectors
\def\erva{{\textnormal{a}}}
\def\ervb{{\textnormal{b}}}
\def\ervc{{\textnormal{c}}}
\def\ervd{{\textnormal{d}}}
\def\erve{{\textnormal{e}}}
\def\ervf{{\textnormal{f}}}
\def\ervg{{\textnormal{g}}}
\def\ervh{{\textnormal{h}}}
\def\ervi{{\textnormal{i}}}
\def\ervj{{\textnormal{j}}}
\def\ervk{{\textnormal{k}}}
\def\ervl{{\textnormal{l}}}
\def\ervm{{\textnormal{m}}}
\def\ervn{{\textnormal{n}}}
\def\ervo{{\textnormal{o}}}
\def\ervp{{\textnormal{p}}}
\def\ervq{{\textnormal{q}}}
\def\ervr{{\textnormal{r}}}
\def\ervs{{\textnormal{s}}}
\def\ervt{{\textnormal{t}}}
\def\ervu{{\textnormal{u}}}
\def\ervv{{\textnormal{v}}}
\def\ervw{{\textnormal{w}}}
\def\ervx{{\textnormal{x}}}
\def\ervy{{\textnormal{y}}}
\def\ervz{{\textnormal{z}}}

% Random matrices
\def\rmA{{\mathbf{A}}}
\def\rmB{{\mathbf{B}}}
\def\rmC{{\mathbf{C}}}
\def\rmD{{\mathbf{D}}}
\def\rmE{{\mathbf{E}}}
\def\rmF{{\mathbf{F}}}
\def\rmG{{\mathbf{G}}}
\def\rmH{{\mathbf{H}}}
\def\rmI{{\mathbf{I}}}
\def\rmJ{{\mathbf{J}}}
\def\rmK{{\mathbf{K}}}
\def\rmL{{\mathbf{L}}}
\def\rmM{{\mathbf{M}}}
\def\rmN{{\mathbf{N}}}
\def\rmO{{\mathbf{O}}}
\def\rmP{{\mathbf{P}}}
\def\rmQ{{\mathbf{Q}}}
\def\rmR{{\mathbf{R}}}
\def\rmS{{\mathbf{S}}}
\def\rmT{{\mathbf{T}}}
\def\rmU{{\mathbf{U}}}
\def\rmV{{\mathbf{V}}}
\def\rmW{{\mathbf{W}}}
\def\rmX{{\mathbf{X}}}
\def\rmY{{\mathbf{Y}}}
\def\rmZ{{\mathbf{Z}}}

% Elements of random matrices
\def\ermA{{\textnormal{A}}}
\def\ermB{{\textnormal{B}}}
\def\ermC{{\textnormal{C}}}
\def\ermD{{\textnormal{D}}}
\def\ermE{{\textnormal{E}}}
\def\ermF{{\textnormal{F}}}
\def\ermG{{\textnormal{G}}}
\def\ermH{{\textnormal{H}}}
\def\ermI{{\textnormal{I}}}
\def\ermJ{{\textnormal{J}}}
\def\ermK{{\textnormal{K}}}
\def\ermL{{\textnormal{L}}}
\def\ermM{{\textnormal{M}}}
\def\ermN{{\textnormal{N}}}
\def\ermO{{\textnormal{O}}}
\def\ermP{{\textnormal{P}}}
\def\ermQ{{\textnormal{Q}}}
\def\ermR{{\textnormal{R}}}
\def\ermS{{\textnormal{S}}}
\def\ermT{{\textnormal{T}}}
\def\ermU{{\textnormal{U}}}
\def\ermV{{\textnormal{V}}}
\def\ermW{{\textnormal{W}}}
\def\ermX{{\textnormal{X}}}
\def\ermY{{\textnormal{Y}}}
\def\ermZ{{\textnormal{Z}}}

% Vectors
\def\vzero{{\bm{0}}}
\def\vone{{\bm{1}}}
\def\vmu{{\bm{\mu}}}
\def\vtheta{{\bm{\theta}}}
\def\va{{\bm{a}}}
\def\vb{{\bm{b}}}
\def\vc{{\bm{c}}}
\def\vd{{\bm{d}}}
\def\ve{{\bm{e}}}
\def\vf{{\bm{f}}}
\def\vg{{\bm{g}}}
\def\vh{{\bm{h}}}
\def\vi{{\bm{i}}}
\def\vj{{\bm{j}}}
\def\vk{{\bm{k}}}
\def\vl{{\bm{l}}}
\def\vm{{\bm{m}}}
\def\vn{{\bm{n}}}
\def\vo{{\bm{o}}}
\def\vp{{\bm{p}}}
\def\vq{{\bm{q}}}
\def\vr{{\bm{r}}}
\def\vs{{\bm{s}}}
\def\vt{{\bm{t}}}
\def\vu{{\bm{u}}}
\def\vv{{\bm{v}}}
\def\vw{{\bm{w}}}
\def\vx{{\bm{x}}}
\def\vy{{\bm{y}}}
\def\vz{{\bm{z}}}

% Elements of vectors
\def\evalpha{{\alpha}}
\def\evbeta{{\beta}}
\def\evepsilon{{\epsilon}}
\def\evlambda{{\lambda}}
\def\evomega{{\omega}}
\def\evmu{{\mu}}
\def\evpsi{{\psi}}
\def\evsigma{{\sigma}}
\def\evtheta{{\theta}}
\def\eva{{a}}
\def\evb{{b}}
\def\evc{{c}}
\def\evd{{d}}
\def\eve{{e}}
\def\evf{{f}}
\def\evg{{g}}
\def\evh{{h}}
\def\evi{{i}}
\def\evj{{j}}
\def\evk{{k}}
\def\evl{{l}}
\def\evm{{m}}
\def\evn{{n}}
\def\evo{{o}}
\def\evp{{p}}
\def\evq{{q}}
\def\evr{{r}}
\def\evs{{s}}
\def\evt{{t}}
\def\evu{{u}}
\def\evv{{v}}
\def\evw{{w}}
\def\evx{{x}}
\def\evy{{y}}
\def\evz{{z}}

% Matrix
\def\mA{{\bm{A}}}
\def\mB{{\bm{B}}}
\def\mC{{\bm{C}}}
\def\mD{{\bm{D}}}
\def\mE{{\bm{E}}}
\def\mF{{\bm{F}}}
\def\mG{{\bm{G}}}
\def\mH{{\bm{H}}}
\def\mI{{\bm{I}}}
\def\mJ{{\bm{J}}}
\def\mK{{\bm{K}}}
\def\mL{{\bm{L}}}
\def\mM{{\bm{M}}}
\def\mN{{\bm{N}}}
\def\mO{{\bm{O}}}
\def\mP{{\bm{P}}}
\def\mQ{{\bm{Q}}}
\def\mR{{\bm{R}}}
\def\mS{{\bm{S}}}
\def\mT{{\bm{T}}}
\def\mU{{\bm{U}}}
\def\mV{{\bm{V}}}
\def\mW{{\bm{W}}}
\def\mX{{\bm{X}}}
\def\mY{{\bm{Y}}}
\def\mZ{{\bm{Z}}}
\def\mBeta{{\bm{\beta}}}
\def\mPhi{{\bm{\Phi}}}
\def\mLambda{{\bm{\Lambda}}}
\def\mSigma{{\bm{\Sigma}}}

% Tensor
\DeclareMathAlphabet{\mathsfit}{\encodingdefault}{\sfdefault}{m}{sl}
\SetMathAlphabet{\mathsfit}{bold}{\encodingdefault}{\sfdefault}{bx}{n}
\newcommand{\tens}[1]{\bm{\mathsfit{#1}}}
\def\tA{{\tens{A}}}
\def\tB{{\tens{B}}}
\def\tC{{\tens{C}}}
\def\tD{{\tens{D}}}
\def\tE{{\tens{E}}}
\def\tF{{\tens{F}}}
\def\tG{{\tens{G}}}
\def\tH{{\tens{H}}}
\def\tI{{\tens{I}}}
\def\tJ{{\tens{J}}}
\def\tK{{\tens{K}}}
\def\tL{{\tens{L}}}
\def\tM{{\tens{M}}}
\def\tN{{\tens{N}}}
\def\tO{{\tens{O}}}
\def\tP{{\tens{P}}}
\def\tQ{{\tens{Q}}}
\def\tR{{\tens{R}}}
\def\tS{{\tens{S}}}
\def\tT{{\tens{T}}}
\def\tU{{\tens{U}}}
\def\tV{{\tens{V}}}
\def\tW{{\tens{W}}}
\def\tX{{\tens{X}}}
\def\tY{{\tens{Y}}}
\def\tZ{{\tens{Z}}}


% Graph
\def\gA{{\mathcal{A}}}
\def\gB{{\mathcal{B}}}
\def\gC{{\mathcal{C}}}
\def\gD{{\mathcal{D}}}
\def\gE{{\mathcal{E}}}
\def\gF{{\mathcal{F}}}
\def\gG{{\mathcal{G}}}
\def\gH{{\mathcal{H}}}
\def\gI{{\mathcal{I}}}
\def\gJ{{\mathcal{J}}}
\def\gK{{\mathcal{K}}}
\def\gL{{\mathcal{L}}}
\def\gM{{\mathcal{M}}}
\def\gN{{\mathcal{N}}}
\def\gO{{\mathcal{O}}}
\def\gP{{\mathcal{P}}}
\def\gQ{{\mathcal{Q}}}
\def\gR{{\mathcal{R}}}
\def\gS{{\mathcal{S}}}
\def\gT{{\mathcal{T}}}
\def\gU{{\mathcal{U}}}
\def\gV{{\mathcal{V}}}
\def\gW{{\mathcal{W}}}
\def\gX{{\mathcal{X}}}
\def\gY{{\mathcal{Y}}}
\def\gZ{{\mathcal{Z}}}

% Sets
\def\sA{{\mathbb{A}}}
\def\sB{{\mathbb{B}}}
\def\sC{{\mathbb{C}}}
\def\sD{{\mathbb{D}}}
% Don't use a set called E, because this would be the same as our symbol
% for expectation.
\def\sF{{\mathbb{F}}}
\def\sG{{\mathbb{G}}}
\def\sH{{\mathbb{H}}}
\def\sI{{\mathbb{I}}}
\def\sJ{{\mathbb{J}}}
\def\sK{{\mathbb{K}}}
\def\sL{{\mathbb{L}}}
\def\sM{{\mathbb{M}}}
\def\sN{{\mathbb{N}}}
\def\sO{{\mathbb{O}}}
\def\sP{{\mathbb{P}}}
\def\sQ{{\mathbb{Q}}}
\def\sR{{\mathbb{R}}}
\def\sS{{\mathbb{S}}}
\def\sT{{\mathbb{T}}}
\def\sU{{\mathbb{U}}}
\def\sV{{\mathbb{V}}}
\def\sW{{\mathbb{W}}}
\def\sX{{\mathbb{X}}}
\def\sY{{\mathbb{Y}}}
\def\sZ{{\mathbb{Z}}}

% Entries of a matrix
\def\emLambda{{\Lambda}}
\def\emA{{A}}
\def\emB{{B}}
\def\emC{{C}}
\def\emD{{D}}
\def\emE{{E}}
\def\emF{{F}}
\def\emG{{G}}
\def\emH{{H}}
\def\emI{{I}}
\def\emJ{{J}}
\def\emK{{K}}
\def\emL{{L}}
\def\emM{{M}}
\def\emN{{N}}
\def\emO{{O}}
\def\emP{{P}}
\def\emQ{{Q}}
\def\emR{{R}}
\def\emS{{S}}
\def\emT{{T}}
\def\emU{{U}}
\def\emV{{V}}
\def\emW{{W}}
\def\emX{{X}}
\def\emY{{Y}}
\def\emZ{{Z}}
\def\emSigma{{\Sigma}}

% entries of a tensor
% Same font as tensor, without \bm wrapper
\newcommand{\etens}[1]{\mathsfit{#1}}
\def\etLambda{{\etens{\Lambda}}}
\def\etA{{\etens{A}}}
\def\etB{{\etens{B}}}
\def\etC{{\etens{C}}}
\def\etD{{\etens{D}}}
\def\etE{{\etens{E}}}
\def\etF{{\etens{F}}}
\def\etG{{\etens{G}}}
\def\etH{{\etens{H}}}
\def\etI{{\etens{I}}}
\def\etJ{{\etens{J}}}
\def\etK{{\etens{K}}}
\def\etL{{\etens{L}}}
\def\etM{{\etens{M}}}
\def\etN{{\etens{N}}}
\def\etO{{\etens{O}}}
\def\etP{{\etens{P}}}
\def\etQ{{\etens{Q}}}
\def\etR{{\etens{R}}}
\def\etS{{\etens{S}}}
\def\etT{{\etens{T}}}
\def\etU{{\etens{U}}}
\def\etV{{\etens{V}}}
\def\etW{{\etens{W}}}
\def\etX{{\etens{X}}}
\def\etY{{\etens{Y}}}
\def\etZ{{\etens{Z}}}

% The true underlying data generating distribution
\newcommand{\pdata}{p_{\rm{data}}}
% The empirical distribution defined by the training set
\newcommand{\ptrain}{\hat{p}_{\rm{data}}}
\newcommand{\Ptrain}{\hat{P}_{\rm{data}}}
% The model distribution
\newcommand{\pmodel}{p_{\rm{model}}}
\newcommand{\Pmodel}{P_{\rm{model}}}
\newcommand{\ptildemodel}{\tilde{p}_{\rm{model}}}
% Stochastic autoencoder distributions
\newcommand{\pencode}{p_{\rm{encoder}}}
\newcommand{\pdecode}{p_{\rm{decoder}}}
\newcommand{\precons}{p_{\rm{reconstruct}}}

\newcommand{\laplace}{\mathrm{Laplace}} % Laplace distribution

\newcommand{\E}{\mathbb{E}}
\newcommand{\Ls}{\mathcal{L}}
\newcommand{\R}{\mathbb{R}}
\newcommand{\emp}{\tilde{p}}
\newcommand{\lr}{\alpha}
\newcommand{\reg}{\lambda}
\newcommand{\rect}{\mathrm{rectifier}}
\newcommand{\softmax}{\mathrm{softmax}}
\newcommand{\sigmoid}{\sigma}
\newcommand{\softplus}{\zeta}
\newcommand{\KL}{D_{\mathrm{KL}}}
\newcommand{\Var}{\mathrm{Var}}
\newcommand{\standarderror}{\mathrm{SE}}
\newcommand{\Cov}{\mathrm{Cov}}
% Wolfram Mathworld says $L^2$ is for function spaces and $\ell^2$ is for vectors
% But then they seem to use $L^2$ for vectors throughout the site, and so does
% wikipedia.
\newcommand{\normlzero}{L^0}
\newcommand{\normlone}{L^1}
\newcommand{\normltwo}{L^2}
\newcommand{\normlp}{L^p}
\newcommand{\normmax}{L^\infty}

\newcommand{\parents}{Pa} % See usage in notation.tex. Chosen to match Daphne's book.

\DeclareMathOperator*{\argmax}{arg\,max}
\DeclareMathOperator*{\argmin}{arg\,min}

\DeclareMathOperator{\sign}{sign}
\DeclareMathOperator{\Tr}{Tr}
\let\ab\allowbreak


\usepackage{hyperref}
\usepackage{url}
\usepackage{graphicx}


\title{ReDER: Related Domain Experience Replay in Deep Reinforcement Learning}

% Authors must not appear in the submitted version. They should be hidden
% as long as the \iclrfinalcopy macro remains commented out below.
% Non-anonymous submissions will be rejected without review.

\author{\textbf{Richard Tang (rtang26), Kenta Yoshii (kyoshii), Raymond Dai (rdai4),
    \& Akash Singirikonda (asingir5)} \\
Brown University\\
Providence, RI 02912, USA \\
\texttt{\{richard\_tang, kenta\_yoshii, raymond\_dai, akash\_singirikonda\}@brown.edu}
}


% \author{Antiquus S.~Hippocampus, Natalia Cerebro \& Amelie P. Amygdale \thanks{ Use footnote for providing further information
% about author (webpage, alternative address)---\emph{not} for acknowledging
% funding agencies.  Funding acknowledgements go at the end of the paper.} \\
% Department of Computer Science\\
% Cranberry-Lemon University\\
% Pittsburgh, PA 15213, USA \\
% \texttt{\{hippo,brain,jen\}}@cs.cranberry-lemon.edu \\
% \And
% Ji Q. Ren \& Yevgeny LeNet \\
% Department of Computational Neuroscience \\
% University of the Witwatersrand \\
% Joburg, South Africa \\
% \texttt{\{robot,net\}@wits.ac.za} \\
% \AND
% Coauthor \\
% Affiliation \\
% Address \\
% \texttt{email}
% }

% The \author macro works with any number of authors. There are two commands
% used to separate the names and addresses of multiple authors: \And and \AND.
%
% Using \And between authors leaves it to \LaTeX{} to determine where to break
% the lines. Using \AND forces a linebreak at that point. So, if \LaTeX{}
% puts 3 of 4 authors names on the first line, and the last on the second
% line, try using \AND instead of \And before the third author name.

\newcommand{\fix}{\marginpar{FIX}}
\newcommand{\new}{\marginpar{NEW}}

\iclrfinalcopy % Uncomment for camera-ready version, but NOT for submission.

\begin{document}


\maketitle



\begin{figure}[htpb]
  \centering
  \includegraphics[width=0.3\textwidth]{imgs/SpaceInvaders.png}
  \caption{Space Invaders, a classic Atari game}
  \label{fig:imgs-SpaceInvaders-png}
\end{figure}

\section{Introduction}

Recent research in deep reinforcement learning has yielded great advancements, with successes from
simple games like Backgammon (\url{https://dl.acm.org/doi/10.1145/203330.203343}) and Atari
(\url{https://arxiv.org/pdf/1312.5602.pdf}) to incredibly complex tasks such as beating Lee Sedol, a
9-dan professional at Go (\url{https://www.nature.com/articles/nature16961}) and multi-step robotics
tasks (\url{https://arxiv.org/pdf/2104.07749.pdf}). However, deep learning models already take
significant resources to train, and reinforcement learning agents worsen this issue by an order of
magnitude. For instance, AlphaGo trained over 72 million matches in 72 hours, using over 6000 TPUs
for self-play simulation, and 64 GPUs and 19 CPUs for parameter updates; in total, training AlphaGo
cost \$35 million. In a model-free environment, DeepMind's Agent57
(\url{https://arxiv.org/abs/2003.13350}) required $10$ billion episodic steps to train.

Clearly, deep reinforcement learning agents are incredibly computationally expensive to train. In
this project, we attempt to alleviate this cost. From experience, humans are relatively adept at
picking up new skills; for instance, proficiency in one first-person shooter game (e.g. Valorant)
would likely transfer to other games with similar play styles and mechanics, such as CS:GO or Apex.
Intuitively, this aligns with our notion of \textit{shared experience}, or experience transfer;
skillsets applicable to one area should theoretically transfer to another similar area, greatly
lessening the initial learning and startup overhead required.

To that end, we first aim to develop a deep reinforcement learning agent---based on current
reinforcement learning literature---capable of playing Atari games, then attempt to employ a
\textit{shared experience buffer} while training to simultaneously learn Atari games with similar
styles; here we choose top-down shooter games, specifically Space Invaders and Demon Attack.



\section{Background}

In classic reinforcement learning (RL)\footnote{OpenAI's ``Spinning Up'' Reinforcement tutorial
provides a great introduction to RL; we adopted parts of their tutorial in this section. Check it
out at \url{https://spinningup.openai.com/en/latest/spinningup/rl_intro.html}}, an \textbf{agent}
interacts with an \textbf{environment}, and attempts to maximize cumulative \textbf{reward}.
Specifically, given a state space $S$ of all valid states in an environment, and an action space $A$
of all valid actions, an agent adopts a
\textbf{policy} \[
  \pi: S\longrightarrow A, \pi(s_t)=a_t
\] that picks an available action $a\in A$ from a current state $s \in S$. In general, state and
action spaces may be either discrete or continuous spaces; moreover, taking an action $a_t$ in a
state $s_t$ can either deterministically or stochastically \textbf{transition} into a state
$s_{t+1}$. With Atari games, both the state and action spaces are discrete, and the environment
transitions is deterministic. A \textbf{reward} value is assigned to these transitions; that is, a
function \[
  R: S\times A\times S\longrightarrow \R,\ R(s_t, a_t, s_{t+1})=r_t
\] determines the reward of a certain transition from one state into another.

The central goal of reinforcement learning is to maximize cumulative reward over an entire episode
in the environment. Because RL agents do not have access to unlimited future rewards, to update and
train our model we turn to the \textbf{finite discounted future reward} function over a time
interval $T$: \[
  R_T(\tau)=\sum_{t'=t}^{T} \gamma^{t-t'}r_t
,\] where $\tau=(s_{t'}, a_{t'}, s_{t'+1}, a_{t'}+1, \ldots)$ is a trajectory of state-action pairs,
and $\gamma\in (0,1)$ is a discount factor on future rewards. Intuitively, we wish to prioritize
current rewards over future rewards; mathematically, we need reward values to converge. Reward
factors into the \textbf{value} of a state, computed by a function \[
  V^{\pi}:S\longrightarrow \R, V^{\pi}(s_t)=v_t
.\] Two functions---the value function and the $Q$ function---are particularly important in
optimizing future reward, and play a central role in reinforcement learning algorithms. They are
given by
\begin{align*}
  V^{\pi}(s)&=E_{\tau\sim \pi}[R(\tau)\mid s_0=s]\\
  Q^{\pi}(s, a)&=E_{\tau\sim \pi}[R(\tau)\mid s_0=s, a_0=a]
.\end{align*} The value function determines the expected reward of starting in a state
$s$ following a certain policy $\pi$, while the $Q$ function determines the expected reward of
starting in a state $s$, taking an action $a$, then following a certain policy $\pi$. Their
respective \textbf{Bellman equations} provide more insight, and format in a more accessible manner
for code implementation:
\begin{align*}
    V^{\pi}(s)&= E_{a\sim \pi, P(s'\mid s,a)}[R(s, \cdot , \cdot )+\gamma V^{\pi}(\cdot )] \\
              &= \max_a \sum_{s'\in S} P(s'\mid s,a)E[R(s,a,s')+\gamma V^\pi(s')] \\
    Q^\pi(s) &= E_{P(s'\mid s,a)}[R(s,a,\cdot)+\gamma V^{\pi}(\cdot )] \\
             &= \sum_{s'\in S} P(s'\mid s,a)[R(s,a,s')+\gamma V^\pi(s')]
.\end{align*}

Determining these functions greatly facilitate RL agents in solving an environment; given an optimal
$Q$ function $Q^*$, an optimal policy $\pi^*$ would simply select the action with the highest $Q$
value in a state $s$; that is, \[
  \pi^*(s)=\max_a Q^*(s,a)
.\] Unfortunately, with many scenarios our environments become too unwieldy to provide exact value
and $Q$ functions; thus, modern reinforcement learning attempts to use function approximators to
estimate these values. As deep learning models become increasingly sophisticated, applications to
reinforcement learning become promising. With the $Q$ function, a neural network (a $Q$-network) may
be used as a non-linear approximator, i.e. $Q(s, a; \theta)\approx Q^*(s,a)$. To train a
$Q$-network, we want to minimize distance from the target $Q$-function: \[
  L_t(\theta_t)=E[(y_t-Q(s,a;\theta_t))^2]
,\] where $y_t=E[r_t+\gamma\max_{a'} Q(s',a';\theta_{t+1})]$ is the target $Q$ function. This is the
\textbf{temporal-difference loss}; that is, the loss between our current $Q$ network and the $Q$
network of the next step, which functions as our target $Q$ network. Because we do not have a ground
truth optimal $Q$ function, we instead train by constantly learning and updating our $Q$-net with
``future'' $Q$-values. From this, one can see how training RL models has extremely high variance,
and is susceptible to model collapse.

To rememdy this, a few modifications are made. To clip error values and reduce variance, we employ
the Huber loss function instead of MSE loss. That is, \[
  L_{\delta}(y_{true}, y_{pred})=\left\{\begin{array}{rl} \frac{1}{2}(y_{true}-y_{pred})^2, &
      ~\text{for}~ \left| y_{true}-y_{pred} \right|  \le \delta \\ \delta(\left| y_{true}-y_{pred}
      \right|-\frac{1}{2}\delta ), &~\text{otherwise}~\end{array}\right.
.\] Intuitively, with larger loss values, we inherently clip them such that no drastic changes are
made to the model (see Figure \ref{fig:imgs-huber_loss-png} for a visualization of its functionality). 

\begin{figure}
  \centering
  \includegraphics[width=0.5\textwidth]{imgs/huber_loss.png}
  \caption{Huber Loss vs. MSE, from Wikipedia (Huber Loss)}
  \label{fig:imgs-huber_loss-png}
\end{figure}

Additionally, Minh et. al. 2015 (\url{https://www.nature.com/articles/nature14236.pdf}) provide
a method of stabilizing variance in the training; because a naive DQN trains solely on the next
step's $Q$-network, variance is extremely high, and the model is susceptible to local minima traps.
To reduce this, a stabilizing $\hat{Q}$ network is introduced to prevent the model from
over-deviating from previous training weights. This $\hat{Q}$ network will share the weights as the
original $Q$ network, except it is not trained; $\hat{Q}$ will only update its values to the
original $Q$ network periodically instead of every step. Thus, its function as an anchor prevents
the training from fluctuating too far from previous steps. Our loss function then becomes \[
  L_t(\theta_t)=E[(y_t-Q(s,a;\theta_t))^2]
,\] where $y_t=E[r_t+\gamma \hat{Q}(s', \max_{a'}(s', a';\theta_{t+1}); \theta_{\hat{Q}})]$. This
architecture of a second stabilizing $\hat{Q}$ target network is called a \textbf{double deep
Q-network}, or \textbf{DDQN}. 

Finally, Wang et. al. 2016 (\url{https://arxiv.org/pdf/1511.06581.pdf}) provide an improvement to
the naive $Q$-network. Instead of learning directly for a $Q$-function approximator, the $Q$-network
is split up into approximating functions $V(s)$ and $A(s)$, or the value function and
\textbf{advantage} function respectively. The advantage function determines the \textit{relative}
value of taking a certain action in a state, rather than the absolute value: \[
  A^\pi(s, a)=Q^\pi(s, a)-V^\pi(s)
.\] This reduces variance by comparatively determining the value of a state, rather than absolutely
predicting it; thus, in extreme states a large $Q$ value is avoided. The ``dueling'' architectures
of the value and advantage function can theoretically learn valuable (and not valuable) states
without having to learn the effect of each action in each state, as the naive $Q$-network is forced
to; this comes as a result of the model learning each separate network, and indirectly predicting
the $Q$-values. The $Q$-function becomes \[
  Q(s, a)=V(s)+\left( A(s,a) - \frac{1}{\left| \mathcal{A} \right| }\sum_{a'\in A}A(s,a') \right) 
.\] This architecture of splitting up the $Q$-network into value $V$ and advantage $A$ streams is
called a \textbf{dueling deep Q-network}, or \textbf{Dueling DQN}. Combined with the DDQN, the final
model is a \textbf{Dueling DDQN} (Figure \ref{fig:imgs-dueling-png}).

\begin{figure}[htpb]
  \centering
  \includegraphics[width=0.7\textwidth]{imgs/dueling.png}
  \caption{Dueling vs. Naive DQN Architecture}
  \label{fig:imgs-dueling-png}
\end{figure}

\section{Methodology}


Our baseline Dueling DDQN agent follows a similar structure to the Mnih et al. 2015 network. An
input image is passed through four convolutional layers: the first has 32 $8\times 8$ filters with
stride $4$, the second has $64$ $4\times 4$ filters with stride $2$, and the third has $64$ $3\times
3$ filters with stride $1$. After some experimentation, we decided to add a fourth convolutional
layer with $1024$ $7\times 7$ filters with stride $1$; this provided more input weights for the
value and advantage streams to work with. The output of the convolutional layers are then split up
into $2$, one for each stream, and passed through a fully-connected dense layer; the value network
outputs a single value, while the advantage network outputs an advantage value for each action in
the action space.

Taking inspiration from stochastic gradient descent, the temporal-difference loss function is
applied in batches, rather than for each individual step. To ensure a more independent and
identically distributed batch of past experiences, we maintain an \textbf{experience buffer} of
previous experiences, from which we sample a batch. We then feed the batch into the loss function
described above, then apply the gradients on our $Q$-network.

This provides the baseline RL agent to train on Atari games; while some modifications were made,
this is mostly in line with methods given in classic Deep RL papers. State representation of shared
experiences is far trickier. At the heart of our learning process is one fundamental question: how
do we convert an experience from one game to another? We intuitively understand that Space Invaders
and Demon Attack have similar mechanics, that learning to dodge bullets in one game will help you
dodge bullets in another, and hitting enemies in both boosts your scores. However, machines have no
such understanding of states, and any architecture used to interpret the screen in one game will
probably fail in the other because of the stark visual differences, animations that play, etc. As
such, we want to develop a state representation such that we can somehow map an experience in one
game to an experience in another. We have attempted several different methods to encourage shared
learning between the two games.

\subsection{Direct Learning}
By far the simplest way to attempt to learn from either experience is to ignore
state representations entirely and train the DQN on experiences as if they came from the same game.
This would involve running two games at once and having a shared experience buffer such that the
model has a chance of learning from either game. This theoretically encourages the state
interpreters---our Dueling DDQN---to focus on the similarities of the two games (movement, shooting,
enemy targeting, etc) and forces it to learn ``core mechanics'' first. In theory, this would make
early training more variant in the beginning but converge faster in both games as neither games get
caught up in small details in each game and masters core mechanics. However, this naive approach
still leaves a lot to be desired: with larger training sets and across more models, more variance is
introduced into the state representations, and the model may struggle to find an underlying
similarity with such differences. RL agents are already highly susceptible to local minima, and
increasing variance only increases the possibility of catastrophic interference and model collapse.
Moreover, although we preprocess the inputs (described in the Experiments section), no flexibility
is provided in general; altering state shapes would completely negate the Dueling DDQN's
convolutional architecture.

\subsection{State Translation}
Another issue with direct mapping results from fundamental differences in games; learning on one
game (e.g. Demon Attack), with different animations and environment structures, may interfere with
another game (e.g. Space Invaders) and the model's interpretation of potent features.  As such,
another intuition is to somehow map one game screen to another. We assume that taking the same
series of actions in one game (ie moving left, right, shooting, and pausing) has a direct
``translation'' in the other game where the same series of actions were taken (Figure
\ref{fig:imgs-StateToState}).

\begin{figure}[htpb]
  \centering
  \includegraphics[width=0.6\textwidth]{imgs/StateToState.png}
  \caption{State Translation Mapping}
  \label{fig:imgs-StateToState}
\end{figure}

However, in practice, machines are not very good at learning this weak translation. A network
trained to map images from one game to the other tends to focus on static, unchanging details (ie
the floors, scores, decorative borders, etc). Any attempt for the network to learn something more
interesting, like the ships' movements, will immediately be squashed by another frame where the ship
is in another region.  As such, this model converges to a local minimum where it is very good at
capturing the static, unmoving objects but cannot map enemies, ships, or moving objects. Most of
either screen is taken up by negative space, and since any loss function would have to give some
form of scalar value to describe differences between two images, a model focusing just on static
objects can have reasonably low loss for every frame. This is a lot easier to converge to than the
complex model that tries to learn the nuances of ship movement. Simply put, one state does not
directly map to another, even if the steps leading up to those two states were identical.

\subsection{Latent Spaces}
Another core idea is to develop a state latent space to learn from rather than learning from
screens. Ideally, some model will encode information about both games to a vector of a common
dimension. This would remove our screen size problem, and games would be training on a much more
simple latent vector rather than complicated screens. More importantly, training a strong latent
space allows our model to learn only from the principal components of the game, which lessens the
likelihood that the model gets distracted by minor details and should stabilize training. As such,
our novel idea was to learn a ``double autoencoder'' - a model capable of autoencoding both space
invaders and demon attack. The model would alternate training between frames of demon attack and
space invaders, tugging the gradient long until the model converges to some latent space capable of
modelling both games. Using these latent spaces allows both models to learn on the same “domain”,
making it easier to map states to each other. However, we have so far been preoccupied with states.
Even if our autoencoder is perfect, who is to say that the latent space for space invaders matches
that of demon attack? Maybe one has the first weight devoted to detecting ship movement while the
other uses it to look at bullets. Learning from the autoencoder here could mean that a useful
experience for one game could be mapping two completely irrelevant frames for another. We realize
that it's not necessarily realistic states we are pursuing but realistic experiences.

\subsection{Delta Mapping}
In order to focus on the changes in experiences, we apply the state ``translation'' model to focus on
changes between states rather than states themselves. For every action, we calculate a delta state
function that subtracts the resulting state from the previous state, which should highlight the
changes in states rather than focusing ons tactic objects. Our results, however, were once again
less than promising. Our distribution is once again relatively static, this time having streaks
where ships frequently appear and hover around (Figure \ref{fig:imgs-DeltaModel-png}).

\begin{figure}[htpb]
  \centering
  \includegraphics[width=0.6\textwidth]{imgs/DeltaModel.png}
  \caption{Delta Mapping of States}
  \label{fig:imgs-DeltaModel-png}
\end{figure}

This is a result of something we are yet to mention about these models: overfitting for early
episodes. Most of the initial training for these methods come before we have a trained Q network. As
such, we are forced to train on random walks, which rarely get far and have very low step counts.
This means our models are optimized to early states, like having the ship in the middle and paying
attention to the first few enemies. Still, while this network also yielded negative results, it did
tell us a lot about the early stages of the game, and provide us a way to interpret the model during
training: even if we add a time value to encourage movement, we see that the local minima model
converges to maps where enemies and the ships typically are.

\subsection{Combination Methods}
Our final method attempts to combine the lessons we learned about state representations. Instead of
learning solely states, we want a method that is able to transform experiences from one game to
experiences in another. Having some form of shared ``experience latent space'' could give us the
ideal situation: a space that is able to codify experiences such that any model can learn from and
apply them. As such, we propose the following architecture (Figure
\ref{fig:imgs-CombinationMethod-png}). 

\begin{figure}[htpb]
  \centering
  \includegraphics[width=0.7\textwidth]{imgs/CombinationMethod.png}
  \caption{Combination of our State Representation Models}
  \label{fig:imgs-CombinationMethod-png}
\end{figure}

There are a few conditions that we want this function to satisfy. First, we want the latent space to
be usable enough to extract experiences from it, so we would want any encoded experience
to be returned by the decoder. Second, we wish to glean the most important
information about these experiences: the rewards and actions. We therefore note that any encoded
experience must return the corresponding action and reward, no matter what decoder is used. This
provides the ``common thread'' between the translations that was lacking in the other latent space
implementation: an encoded experience should be decoded to a similar valid experience. Finally, we
want visual similarity: that is, decoded outputs should look like actual states and changes in
states. We achieve this through the use of a discriminator, which tries to determine which
experiences are real and fake (Figure \ref{fig:imgs-Discriminator-png}). 

\begin{figure}[htpb]
  \centering
  \includegraphics[width=0.7\textwidth]{imgs/Discriminator.png}
  \caption{Discriminator aspect of the architecture}
  \label{fig:imgs-Discriminator-png}
\end{figure}

As such, we note that our final structure has something similar both to autoencoders and generative
adversarial networks (Figure \ref{fig:imgs-VAE-png}):

\begin{figure}[htpb]
  \centering
  \includegraphics[width=0.7\textwidth]{imgs/VAE2.png}
  \caption{Variational Auto-encoder aspect}
  \label{fig:imgs-VAE-png}
\end{figure}

Since the latent spaces are common to all games, it provides a proper replacement to raw input
screens as our state representation. Unfortunately, the model is a bit too involved to train on our
current hardware; it takes a minute to train on a singular step of the model given all the gradients
that need to be optimized. However, we have the full model coded, although it is questionable
whether the model will see any gains over just normal training. The entire translation process might
lose some of the ``core gameplay'' advantage that our first method promised, and it may end up
generating the same experiences over and over, which will not improve training.



\section{Experiments}

Before training our models, we first apply some important preprocessing steps on the Atari
environments\footnote{Many of these techniques were first introduced and provided by OpenAI and
DeepMind. The helper functionality was primarily adopted from OpenAI's baseline preprocessing
methods; see
\url{https://github.com/openai/baselines/blob/master/baselines/common/atari_wrappers.py}}. To
facilitate training, we skip every fourth frame; previous RL literature (e.g. Mnih et. al. 2015)
found no detrimental effects on training, while speeding up the process. Then, to reduce the
dimensionality of the image and lessen redundancy in weights, each environment image is first
converted from RGB to grayscale, then downsampled from $216\times 160$ to $84\times 84$. Finally, an
optional state-stacking mechanism is provided, where four consecutive frames are stacked on top of
each other. This provides the model with a way to learn the current trajectory of the environment's
state (Figure \ref{fig:imgs-AtariGames-png}). 

\begin{figure}[htpb]
  \centering
  \includegraphics[width=0.8\textwidth]{imgs/AtariGames.png}
  \caption{Post-preprocessing Atari Images}
  \label{fig:imgs-AtariGames-png}
\end{figure}

Following pre-processing, we train the model. To ensure a proper exploration-exploitation trade-off
and lower the possibility of model collapse through optimizing for a local minima, we employ an
$\varepsilon$-greedy policy, where we explore a random action with probability $\varepsilon$, and
follow the max $Q$-value output from our model with probability $1-\varepsilon$. The $\varepsilon$
values are linearly annealed, with the different values listed below. When calculating the loss
function and applying gradients, we deviate slightly from the strategy proposed in Mnih et. al.
2015; instead of backpropagating on every step, we instead learn every few (listed below)
steps; we found that this lowers potential for model collapse in the beginning, when our model may
be optimizing for a random (and wrong) target $\hat{Q}$ network. Additionally, while the Mnih
architecture updates the target network every $10000$ steps, we choose a lower number (listed below)
to avoid local minima. We choose an Adam optimizer over RMSProp optimizer; at the time of the
original Mnih architecture, the Adam optimizer had not yet been released, and generally for deep
learning models the Adam optimizer provides a more robust optimization function. The complete
hyperparameters of our models are listed below (see Table \ref{table:hyperparameters}).

\begin{table}[htpb]
  \centering
  \caption{Hyperparameters}
  \begin{tabular}{||c | c | c||}
    \hline
   & Base-s1 & Base-s2 / Shared \\
   \hline\hline
    Learning rate & $0.00025$ & $0.00035$ \\
    \hline
    $\varepsilon$-start & $0.95$ & $0.99$ \\
    \hline
    $\varepsilon$-end & $0.05$ & $0.02$ \\
    \hline
    $\varepsilon$-steps & $500000$ & $500000$ \\
    \hline
    $\gamma$ & $0.99$ & $0.995$ \\
    \hline
    Num episodes & $1000000$ & $1000000$ \\
    \hline
    Buffer size & $10000$ & $20000$ \\
    \hline
    Batch size & 16 & 16 \\
    \hline
    Learning frequency & 20 & 10 \\
    \hline
    Target update & 2000 & 2500 \\
    \hline
    Stacked frames & 1 & 4\\
    \hline
  \end{tabular}
  \label{table:hyperparameters}
\end{table}

The experience buffers are initialized by randomly sampling \texttt{BUFFER\_SIZE} states from the
environments; for the shared model, the buffer size is doubled, and states are sampled from both
environments. Due to computational limits, we only trained our models over $500000$ steps, with
evaluation benchmarking every $10000$ steps. With the single baseline models, we train using one
OpenAI gym environment, either Space Invaders or Demon Attack. With the shared model, we train using
two OpenAI gym environments, Space Invaders and Demon Attack; to facilitate the double environment
nature, the model architecture alternates between the two environments, sampling stacked frames one
at a time. When training, we lowered the batch size from the recommended $128$ to $16$, due to more
computational limits imposed during training.


\section{Results}

Overall, due to the limits imposed by time constraints and computational power, we were only able to
train our model for $600000$ steps, significantly lower than the recommended $4000000$ steps (a
common benchmark across RL literature to successfully beat one level of Space Invaders); as a
result, our graphs are less than optimal. However, fascinating (and promising!) results still
appeared.

\subsection{Dueling DDQN Metrics}

We start by analyzing the Dueling DDQN model's performance on Space Invaders. The key difference
between \texttt{base-s1} and \texttt{base-s2} lie in frame stacking; \texttt{base-s1} does not stack
frames, while \texttt{base-s2} stacks four frames on top of each other. Intuitively, we would
suspect that \texttt{base-s2} performs better, as the convolutional model learns consecutive state
transitions. Indeed, this is somewhat corroborated by the loss, $Q$-values, and rewards.

\begin{figure}
    \centering
    \begin{minipage}{0.45\textwidth}
        \centering
        \includegraphics[width=0.9\textwidth]{imgs/loss-base1.png} % first figure itself
        \caption{\texttt{base-s1} Losses}
        \label{fig:loss-base1}
    \end{minipage}\hfill
    \begin{minipage}{0.45\textwidth}
        \centering
        \includegraphics[width=0.9\textwidth]{imgs/loss-base2.png} % second figure itself
        \caption{\texttt{base-s2} Losses}
        \label{fig:loss-base2}
    \end{minipage}
\end{figure}

For both loss values (Figures \ref{fig:loss-base1} and \ref{fig:loss-base2}), they appear relatively
stagnant, with \texttt{base-s2} showing slight increases. While this doesn't fit our usual intuition
of losses, the graphs do appear sensible given the consideration of RL and our lower training steps.
With temporal-difference loss, we compute not the difference between a ground-truth value and our
predicted value, but rather the difference between our current predicted value and the predicted
value of the next state. As a result, until our model is well trained, the values of loss should
remain relatively stagnant; especially given the number of steps needed ($4000000$) versus the
number of steps taken ($600000$), as well as the high complexity of Space Invaders as a whole, it
would make sense that the model still has a lot to learn (Note: the losses for the \texttt{base-s1}
model start at $300000$ rather than $0$, since we unfortunately did not set up metrics logging until
halfway through training).

To verify that learning is actually occurring, we inspect the $Q$-values of the models over time
(Figures \ref{fig:q-base1} and \ref{fig:q-base2}).
\begin{figure}
    \centering
    \begin{minipage}{0.45\textwidth}
        \centering
        \includegraphics[width=0.9\textwidth]{imgs/q-base1.png} % first figure itself
        \caption{\texttt{base-s1} Q values}
        \label{fig:q-base1}
    \end{minipage}\hfill
    \begin{minipage}{0.45\textwidth}
        \centering
        \includegraphics[width=0.9\textwidth]{imgs/q-base2.png} % second figure itself
        \caption{\texttt{base-s2} Q values}
        \label{fig:q-base2}
    \end{minipage}
\end{figure}

Here we see rather promising results. We evaluate the sum of the $Q$-values over a set collection of
states over time, and see how the model values each state. As such, our intuition suggests that over
time, as the model learns the benefits of entering each state, the $Q$-values in the states should
increase. Our models demonstrate this feature; over time, we see a steady increase in the sum of the
$Q$-values. Interestingly, potentially due to the robustness modifications provided by the Dueling
Double DQN structure and the modified loss function, our models appear to successfully avoid
becoming stuck in local minima during training. In \texttt{base-s2}'s $Q$-values, even though they
initially decrease as the model explores detrimental states (note the small dip at the beginning),
the model recovers and manages to properly evaluate $Q$-values.

Finally, we see the results over time of each model (Figures \ref{fig:rewards-base1} and
\ref{fig:rewards-base2}).
\begin{figure}
    \centering
    \begin{minipage}{0.45\textwidth}
        \centering
        \includegraphics[width=0.9\textwidth]{imgs/rewards-base1.png} % first figure itself
        \caption{\texttt{base-s1} Rewards}
        \label{fig:rewards-base1}
    \end{minipage}\hfill
    \begin{minipage}{0.45\textwidth}
        \centering
        \includegraphics[width=0.9\textwidth]{imgs/rewards-base2.png} % second figure itself
        \caption{\texttt{base-s2} Rewards}
        \label{fig:rewards-base2}
    \end{minipage}
\end{figure}

Here, although progress is slow, we can see a clear trend. Over time, the models gradually learn
trends within the games, and how to utilize certain aspects. A qualitative analysis here provides
better insights into the model's progress. At the start of training, inspection of the model's
behavior is similar to what we expect (see GIFs in action here: \url{https://imgur.com/a/m6fDYD0}).
The model generally gravitates towards the left shield of the environment state, as that has
provided the best temporary rewards as the model gradually explores the state space. However, the
model has no insights on the importance of prioritizing the Mothership, the benefits of exploration
away from the left side, and general dodging of the shots; it is truly just a random model.

However, over time the model gradually grows its knowledge of the state space, and learns certain
important features (see GIFs in action here: \url{https://imgur.com/a/gzdgcw3}). First, the model
starts to explore elsewhere; once the first shield is gone, there is no point in hovering around
there, so the model grows the ability to explore other aspects of the field. Indeed, in the last
example, the model moves all the way to the right-most side. More importantly, the model learns the
importance of dodging shots; although not overly effective and sometimes sporadically getting hit,
the model generally senses the negative effects of the shots, and makes an effort to dodge them.
Perhaps most impressively, the model learns the importance of prioritizing the Mothership. Whenever
a Mothership appears, the model makes a concerted effort to move toward the Mothership; while not
always successful (sometimes, the model runs directly into shots), the model understands the reward
benefits, and sometimes impressively is able to hit it (not once, but twice in the last example). 

Over our $600000$ training steps, we achieved a maximum reward of $790$ (bottom GIF), one that we're
very pleasantly surprised with, given the warning of $4000000$ steps needed to pass the level.
Perhaps more impressively, the model managed this in not three lives, but one single life. Given
more time, we would love the explore the capabilities of the model after millions of training steps.

\subsection{Shared Model}

Here we first inspect the ability of the state representation model to properly map values.
Unfortunately, our results here were far from stellar. As it seems, the loss values for both the
State-to-State Translations and Delta Mappings did not provide promising results (Figures
\ref{fig:s2s-loss} and \ref{fig:delta-loss}). 

\begin{figure}
    \centering
    \begin{minipage}{0.45\textwidth}
        \centering
        \includegraphics[width=0.9\textwidth]{imgs/S2S-loss.png} % first figure itself
        \caption{State-to-State Translation Loss}
        \label{fig:s2s-loss}
    \end{minipage}\hfill
    \begin{minipage}{0.45\textwidth}
        \centering
        \includegraphics[width=0.9\textwidth]{imgs/Delta-loss.png} % second figure itself
        \caption{Delta Mappings Loss}
        \label{fig:delta-loss}
    \end{minipage}
\end{figure}

We note that the state-to-state model converges on a local minima; it finds it far more rewarding
to model just the static elements of each game than to actively try to represent changes in the
model. Any attempt at modeling moving ships are likely undone by future layers, making it extremely
difficult to converge further. We see that loss no longer decreases and stays relatively constant.

Like the state to state, the delta mapping model converges to a local minima and does not decreases
its losses very much. The difference is that it converges on a static image of what appears to be
probability distributions of the ship and enemies; clear white streaks can be seen where enemies
typically appear and where the ship moves around. Again, it remains static with no decreasing loss.

Due to computational constraints, we were unable to provide meaningful results for either the latent
spaces or combination methods. Due to the sheer intensity of especially the combination methods, our
model trained a measly $10$ steps over the course of an entire hour; in such a small time frame, no
remotely reasonable results were possible.

Personal time constraints also limited our ability to integrate our state mappings with the rest of
the model; as a result, we could only perform training on a naive, direct learning state
representation (i.e. no states were translated between games). Despite this, our results are rather
optimistic:

\begin{figure}
    \centering
    \begin{minipage}{0.45\textwidth}
        \centering
        \includegraphics[width=0.9\textwidth]{imgs/loss-joint-da.png} % first figure itself
        \caption{Joint Model (DA) Loss}
        \label{fig:loss-joint-da}
    \end{minipage}\hfill
    \begin{minipage}{0.45\textwidth}
        \centering
        \includegraphics[width=0.9\textwidth]{imgs/loss-joint-si.png} % second figure itself
        \caption{Joint Model (SI) Loss}
        \label{fig:loss-joint-si}
    \end{minipage}
\end{figure}

We note that the loss functions (Figures \ref{fig:loss-joint-da} and \ref{fig:loss-joint-si}) behave
rather similarly across all the models we created; it seems to decrease a bit towards the beginning
but flatten out in the middle of training. There also seems to be a high variation in loss, as shown
by how many spikes are in the graphs.

However, what is more interesting about the joint models is their Q functions (Figures
\ref{fig:q-joint-da} and \ref{fig:q-joint-si}):

\begin{figure}
    \centering
    \begin{minipage}{0.45\textwidth}
        \centering
        \includegraphics[width=0.9\textwidth]{imgs/q-joint-da.png} % first figure itself
        \caption{Joint Model (DA) Q Values}
        \label{fig:q-joint-da}
    \end{minipage}\hfill
    \begin{minipage}{0.45\textwidth}
        \centering
        \includegraphics[width=0.9\textwidth]{imgs/q-joint-si.png} % second figure itself
        \caption{Joint Model (SI) Q Values}
        \label{fig:q-joint-si}
    \end{minipage}
\end{figure}


In terms of Q functions, both joint models seem to spike in the beginning, jumping from a negative
to positive value; this exhibits the same resilience with Q-values as the individual model.
Afterwards, they see steady linear growth, which is not uncommon among our models; perhaps what is
surprising is that even with the alternating environments, the model is capable of generally
increasing its $Q$-value estimations.

The reward function trends provide the most notable results (Figures \ref{fig:rewards-joint-da} and
\ref{fig:rewards-joint-si}):

\begin{figure}
    \centering
    \begin{minipage}{0.45\textwidth}
        \centering
        \includegraphics[width=0.9\textwidth]{imgs/rewards-joint-da.png} % first figure itself
        \caption{Joint Model (DA) Rewards}
        \label{fig:rewards-joint-da}
    \end{minipage}\hfill
    \begin{minipage}{0.45\textwidth}
        \centering
        \includegraphics[width=0.9\textwidth]{imgs/rewards-joint-si.png} % second figure itself
        \caption{Joint Model (SI) Rewards}
        \label{fig:rewards-joint-si}
    \end{minipage}
\end{figure}

The Joint Space-Invaders model hits about 60 reward value with 800 episodes, whereas \texttt{base2}
only hit that mark at about $1250$ episodes (a bit of data was lost for \texttt{base1}, making it
seem like it converges faster than it does). On the other hand, Demon Attack averaged at about $40$
reward value at 400 episodes, whereas base Demon Attack was around $50$. We see slightly faster and
slower convergences respectively, though it is unclear what would happen if the model trained even
longer or, if it is within a margin of error. Either way, we remain cautiously optimistic about the
potential results of our model; we see definitive strategy changes between the two models, and
believe that shared experience probably trains just as well, if not better, than standard baseline
models. More tests are necessary to test the method's efficacy, but visually we can see some shared
knowledge informing decisions on both games.


\section{Challenges}

Overall, our project presented significantly more challenges than expected.

Perhaps the first we encountered was the sheer complexity of reinforcement learning research,
combined with our naivete and lack of appreciation of its scope. As a result, the focus of our
project shifted significantly over time. Our original idea came from the Actionable Models paper
(linked in the introduction); with a combination of hindsight experience replay and actionable
models on offline reinforcement learning datasets collected from robotic arms, we wished to
re-implement key aspects of the paper and hopefully modify the architecture to apply to board games,
such as Chess and Shogi. 

Unfortunately, as we quickly realized, our knowledge of reinforcement learning, robotics, and board
games were all severely lacking. As we continued research on the topic, comprehending the novel
architectures in the papers themselves proved extremely difficult, let alone implementing them.
After a fruitless week of bashing our heads against robotics and board games research papers, we
decided to shift our idea toward an Atari-focused RL model.

This proved a little easier to digest, but still presented major problems. At this point, we began
the RL section of the course, which aided in our understanding of the papers; we understood key
concepts from Deep Q-Learning and Policy Gradient Learning methods, and better understood how the
models trained. However, the sheer complexity of reinforcement learning papers still existed.
Moreover, while most RL papers provided pseudocode for the implementations of their algorithms, many
aspects were greatly unspecified, and converting the ideas into tangible code proved a great
challenge. Even though some online resources were provided, we found it extremely difficult to
parse, due to either poor/missing documentation, or significant optimizations that resulted in
almost unparseable code (i.e. OpenAI's baseline implementations); moreover, the vast majority of
models existed in PyTorch, while our familiarity lied completely in the domain of Tensorflow.

Here too emerged the second major problem: computational resources. Although we began implementing
separate baseline RL models in an attempt to form a baseline, it proved exceedingly difficult to
verify the results of our implementations. Because the models don't provide that many meaningful
metrics, and training steps often vary greatly in their success, we were unable to detect bugs in
our implementation (either the loss function or train function, usually), wasting significant hours
on revising our models.

Even after we formulated a proper model, we lacked the computational resources to train each to
their full extent; despite multiple attempts to set up a Google Cloud Compute engine to train our
agents, we were unable to optimize the benefits of a GPU, and were forced to train completely
locally. As a result, the maximum amount of time-steps that we could feasibly devote to each model
happened to be around $600K$ steps, which fell significantly below the suggested $4M$ steps. The
baseline and joint models thus did not fully train, yielding subpar visualizations.

Moreover, our formulated state representations ended up being significantly more complex than
expected; although we could train the naive state representations (i.e. direct learning,
state-to-state translations), they didn't end up yielding positive results, so we had to revamp our
model architecture with additional steps. The final combination method we formulated ended up
suffering the fate of lacking computational resources; when attempting to train our behemoth on the
environment, we only trained on $10$ steps in an entire hour. Without significantly more compute,
training and testing the model proved impossible. Thus, we were forced to default to training the
joint model on only the naive direct mapping on shared states. Although our results were slightly
promising, the sheer complexity and requisite compute meant that we were unable to acquire
substantive evaluations on the efficacy of our models.

Finally, while it did not prove an insurmountable bottleneck, our relative inexperience with
building end-to-end deep and reinforcement learning systems in general proved a major hurdle. While
we had general experience with implementing certain Tensorflow features from the homework
assignments, we did not know how to construct support code and integration architecture from
scratch. For instance, we focused primarily on getting a model up and running to train; however, we
had not set up appropriate checkpoint, visualization, logging, and metrics architectures to properly
evaluate our functionality. Especially in Deep Q-Networks, where temporal-difference loss doesn't
provide meaningful results until very late into the training process (or even ever), logging the
loss values is not particularly useful. More importantly, we did not thoroughly investigate
image/GIF storage nor saving model checkpoints at first, which resulted in losing the progress on
hundreds of thousands of steps over a few models.


\section{Reflection}

Although we did not manage to completely implement the shared state representation originally
proposed, we are still generally optimistic on the results of our experiments. Once we recognized
the sheer complexity and instability of reinforcement learning, our baseline and target goals
reflected the desire to formulate a working Deep Q-Network to adequately play the Atari games (here,
specifically Space Invaders). To that end, we met our base and target goals: we originally set out
simply to replicate results from the seminal Mnih et al. 2013 Deep Q-Network, and hopefully extend
some basic state representations. As we researched more, we discovered the Dueling Double Deep
Q-Network, and set that as a target goal to implement after recognizing the relative increase in
complexity compared to the naive DQN.

Our baseline DQN models also performed significantly better than we had expected; after seeing the
suggested $4M$ steps to train the model to success on Space Invaders, we were not optimistic at all
about the ability of our model to achieve moderate success on the Atari game. Despite that, after
just around $350K$ steps, our model demonstrated a great performance on a single episodic life,
reaching $790$ points! More importantly though, our model appeared to learn key mechanics of the
game. After around $200K$ steps, our model began to learn to explore beyond the left-most shield.
Moreover, the model's dodging capabilities proved significantly more non-random than we had
expected; and finally, the prioritization of the Mothership was a pleasant surprise.

Given more time, we would have liked to explore more reinforcement learning agents and attempted
different models over a more reasonable number of iterations (e.g. $1M-2M$). For instance, we did
not have the time or compute to explore shared experiences with policy-gradient methods; especially
given their current prominence in RL literature, a more thorough investigation and comparison of
Deep Q-Learning and Policy Gradient methods would have been a worthwhile experience. As mentioned
before, further explorations into more complicated shared domain experience buffers during training
would be insightful as well.

On a personal level, this final project has been immensely beneficial to our growth as both deep
learning engineers/programmers and researchers. After undergoing many iterations and setbacks
throughout the process, we've learned to appreciate the complexity of many recent advancements in
DL/RL research and the difficulty in turning theoretical pseudocode implementations into concrete
models. Moreover, we've developed important skills in setting up and executing a deep learning
system from scratch, and cultivated proper deep learning (and general) programming practices, from
frequent checkpoints and logging to proper visualization and model presentation techniques.

After struggling to no avail with many research papers, we also feel more confident in assessing the
state-of-the-art literature in deep/reinforcement learning, understanding foundational models and
architectures, and most importantly modifying and tinkering with current implementations.  Novel
papers no longer seem impossibly daunting; after some effort, we now feel capable of demystifying
key concepts and turning them into tangible implementations. 

In all, the entire process has been greatly rewarding and fascinating, and after the final project
we would love to continue exploring various RL architectures, playing with them, and (hopefully!)
contributing meaningful results to the scientific community.


\section{Conclusion}

Recent advancements in reinforcement learning give promising signals for the growth of the field;
many efficient and robust models have been developed to tackle various challenges, from sandbox (yet
still fundamentally significant) environments like Atari and board games, to real and meaningful
technological developments, such as robotics and self-driving cars. In each of these instances,
however, exorbitant amounts of training and evaluation were required, resulting in prohibitively
long training times for each individual model.

In this project, we sought first to re-implement---with minor adjustments---the foundational
reinforcement learning agent of Dueling Double Deep Q-Networks as an educational experience in
state-of-the-art reinforcement learning models. Then, we researched potential models to alleviate
the training time through shared experiences, and explored architectures to generalize state
representations into a common latent space. We ran initial training steps on a joint experience
buffer between two similar environments---Atari's Space Invaders and Demon Attack games---to test
the validity of a naive shared experience; while incomplete, we show cautiously optimistic learning
results, providing a proof-of-concept possibility with even a naive implementation. With further
formulation and benchmarking, we hope that shared experiences may be used in different environments
such as robotics to alleviate training time and aggregate learning of similar tasks.


\end{document}
